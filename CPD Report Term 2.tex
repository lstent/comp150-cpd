\documentclass{scrartcl}

\usepackage[hidelinks]{hyperref}
\usepackage[none]{hyphenat}
\usepackage{setspace}
\doublespace

\usepackage{amsmath}

\title{CPD Report}

\subtitle{COMP150 - CPD}

\author{1506919}

\begin{document}

\maketitle

\section*{Introduction}

In this paper I will discuss five key skills, taken from my continuing professional development report for year 1: term 2, that I need to improve on to succeed on my course. Also by using SMART targets as a guide, I will first state the skill, and why it is important to increase my ability, not just for my course, but in terms of my future career as well. Secondly how I plan to improve that skill and how long I expect it to take me to reach my goals.

\section*{Reading/ remembering C++ code and Unreal blueprints}

I have a lot of difficulty with remembering and reading other people C++ code and Unreal blueprints, at the start of semester 2 while going through video tutorials on the Pluralsight website to help me understand C++ coding for comp130 worksheets, I started taking note of code and explaining what it does, which was really good help, but as the semester went on, I gradually stopped. 

This is important for me to improve on to make classes easier in the future, and also I need a good understanding of C++ and blueprints to become a programmer for a AAA studio, I will watch video tutorials and take notes again as this helped me a lot, I will do this over summer when I have no assignments and plenty of spare time to learn, I will aim to watch at least one C++ tutorial a week.

For Unreal blueprints I can watch videos on Pluralsight and the Unreal website I also have test games stored on my hard drive where I practice making AI behavior blueprints, health bar widgets, etc, which I can look at if I need to do that blueprint or similar blueprints for assignments or group games, which I shall continue and do more of over the summer break, making sure to make one new working blueprint every week.

\section*{Teamwork}

I would describe myself as a very introverted worker, meaning I don't talk very much in team meetings, I'd prefer to be told what to do then offer my opinion on ideas. I also find it difficult to make bad comments about peoples work, I will try to be as gentle as possible with small criticisms even if really don't like it or think it should be done differently, which can make me feel like I'm not part of the team. 

This is important to work on for university as it might make me feel more valued within the group, enjoy my work more and increase the amount of energy I put into a piece of coding or blueprint. If I was working for a AAA games studio although they may not ask for my opinion, it might help me become more outgoing within team meetings and be able to say if I'm having trouble with a certain task.

From now until the start of the summer break I will comment on at least one piece of work from my game group a week either positive or negative and hopefully get more used to speaking out, then when we get into our new groups for the second year I will gradually comment more.

\section*{Prioritising Assignments}



\section*{Agile Methodologies}



\section*{Spelling and punctuation}



\section*{Conclusion}



\end{document}